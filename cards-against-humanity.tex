\documentclass[a4paper,parskip]{scrartcl}
\usepackage[top=10mm,bottom=10mm,left=15mm,right=15mm]{geometry}
\usepackage{tikz}
%\usepackage{pifont}
%\usepackage{anttor}
\usepackage[sfdefault]{ClearSans}
\usepackage[utf8]{inputenc}
\begin{document}
\pgfmathsetmacro{\cardwidth}{6.2}
\pgfmathsetmacro{\cardheight}{8.8}
\pgfmathsetmacro{\stripwidth}{0.6}
\pgfmathsetmacro{\strippadding}{0.1}
\pgfmathsetmacro{\textpadding}{0.1}
\pgfmathsetmacro{\ruleheight}{0.15}
\pgfmathsetmacro{\cardsymbolwidth}{0.25}
\pgfmathsetmacro{\cardsymbolheight}{0.4}

\setlength{\baselineskip}{16pt}

\newcommand{\titleleft}{CARDS AGAINST HUMANITY}
\newcommand{\underl}{\rule{2cm}{.5pt}}

\newcommand{\blackcard}[2] {\begin{tikzpicture}%
    \draw (0,0) rectangle (\cardwidth cm,\cardheight cm);
        \fill[black] (0, 0) rectangle (\stripwidth,\cardheight) node[rotate=90,above left,white,font=\large] {
            \titleleft
        };
        \node[fill=black, draw=white, minimum height = \cardsymbolwidth cm, minimum width = \cardsymbolheight cm, anchor=east, rotate=15] at (4*\stripwidth/5, \stripwidth/2) { } ;
        \node[fill=white, draw=black, minimum height = \cardsymbolwidth cm, minimum width = \cardsymbolheight cm, anchor=east, rotate=0] at (4*\stripwidth/5, \stripwidth/2) { } ;
        \node[fill=white, draw=black, minimum height = \cardsymbolwidth cm, minimum width = \cardsymbolheight cm, anchor=east, rotate=-20] at (4*\stripwidth/5, \stripwidth/2) { } ;

        \node[text width=(\cardwidth-\strippadding-\stripwidth-2*\textpadding-0.3)*1cm,below right, anchor=west] at (\strippadding+\stripwidth+\textpadding,\cardheight*1/2-\textpadding) {
            {\huge #2\par}
        };
        \node[anchor=south east] at (\cardwidth, 0) {\scriptsize #1};
\end{tikzpicture}}
\newcommand{\whitecard}[2] {\begin{tikzpicture}%
    \draw (0,0) rectangle (\cardwidth cm,\cardheight cm);
        \draw[gray] (0,0) rectangle (0\stripwidth,\cardheight0) node[rotate=90,above left,black,font=\large] {\titleleft};
        \node[fill=black, draw=black, minimum height = \cardsymbolwidth cm, minimum width = \cardsymbolheight cm, anchor=east, rotate=15] at (4*\stripwidth/5, \stripwidth/2) { } ;
        \node[fill=white, draw=black, minimum height = \cardsymbolwidth cm, minimum width = \cardsymbolheight cm, anchor=east, rotate=0] at (4*\stripwidth/5, \stripwidth/2) { } ;
        \node[fill=white, draw=black, minimum height = \cardsymbolwidth cm, minimum width = \cardsymbolheight cm, anchor=east, rotate=-20] at (4*\stripwidth/5, \stripwidth/2) { } ;

        \node[text width=(\cardwidth-\strippadding-\stripwidth-2*\textpadding-0.3)*1cm,below right, anchor=west] at (\strippadding+\stripwidth+\textpadding,\cardheight*2/3-\textpadding) {
            {\huge #2\par}
        };
        \node[anchor=south east] at (\cardwidth, 0) {\scriptsize #1};
\end{tikzpicture}}

\whitecard{FSK18}{Käse}%
\whitecard{FSK18}{Eicheln}%
\blackcard{FSK18}{Ich habe \underl nie wirklich verstanden, bis ich auf \underl gestoßen bin.}%


\end{document}
